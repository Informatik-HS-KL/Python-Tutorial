% ----------------------------------------------------------------------------------------------
% Vorlage für Dokumentation auf LaTeX-Basis im Projekt InfoStuDi
% --------------------------------------------------------------------------------------------
% utf-8 verwenden, aber der Vorlage vorgaukeln, ansinew würde verwendet werden. Quick-fix um mit
% Manfred Brills mb.sty kompatibel zu sein
\makeatletter
\newcommand{\dontusepackage}[2][]{%
  \@namedef{ver@#2.sty}{9999/12/31}%
  \@namedef{opt@#2.sty}{#1}}
\makeatother
\usepackage[utf8]{inputenc}
\dontusepackage[ansinew]{inputenc}
\usepackage{mb}
\usepackage{mbmath}
\usepackage{textcomp}
% Array-Paket für mehr Kontrolle der Tabellen
\usepackage{array}
% ifthen für Ein- und Ausblenden der Lösungen.
\usepackage{ifthen}
% Paket für Postscript Pi-Fonts
\usepackage{pifont}
% PSTricks
%\usepackage{pstricks}
% Paket für das einfache Umdefinieren von Listen
\usepackage[shortlabels]{enumitem}
% Alphabetischer Index
\usepackage{imakeidx}
\makeindex[title=Index,columns=2,options=-s german,intoc]
% Header und Footer mit KomaScript
\usepackage[automark, headsepline]{scrlayer-scrpage}
%
% Header für Werkzeuge, Begriffe aus dem Software-Management
\input{variablen}
% Farben
\input{colors}
% Koordinatensysteme und andere Makros
\input{coordinateSystems}
% Standardverzeichnis für das Basisverzeichnis der Bilder
%
\newcommand{\imagePath}{./images}
%
\raggedbottom
\setlength{\parskip}{2.0ex}
\setlength{\parindent}{0.0cm}
%% Verhindert Schusterjungen und Hurenkinder
\clubpenalty = 10000
\widowpenalty = 10000 \displaywidowpenalty = 10000
%
\setcounter{secnumdepth}{2}
\setcounter{tocdepth}{2}
%
\pagestyle{scrheadings}
\clearscrheadfoot
% Thema des Dokuments in die Kopfzeile
\ihead{\headmark}
\ohead[]{\pagemark}
\chead{}
\pagestyle{scrheadings}
% Abstände zwischen Caption und Bild/Tabelle
\setlength\abovecaptionskip          {0.4em}
\setlength\belowcaptionskip          {0.2em}
% Anteil der Grafiken höher auf jeder Seite!
\renewcommand{\textfraction}{0.001}
\renewcommand{\topfraction}{0.99}
% Literatur-Stil
\bibliographystyle{geralpha}
% Listingspaket
\usepackage{listings}
\lstloadlanguages{Java}
\lstset{language=Java}
\definecolor{lstback}{gray}{0.85}
\lstset{backgroundcolor=\color{lstback}}
\lstset{extendedchars=true}
\lstset{showstringspaces = false}
\lstset{basicstyle = \ttfamily \small}
%% listings mit listings.sty
%
% Kommando für den Flattersatz bei nebeneinander liegenden Abbildungen
\newcommand{\flatter}{\setlength{\rightskip}{0pt plus 2cm}}
%
% Schritte in einer Aufzählung, dafür einen Zähler (schritt) und die Umgebung
% schritte definieren.
\newcounter{schritt}
\newenvironment{schritte}%
{\begin{list}%
{Schritt \arabic{schritt}:}%
{\usecounter{schritt}\settowidth{\labelwidth}{Schritt 1:}%
\setlength{\leftmargin}{\labelwidth}\addtolength\leftmargin{\labelsep}%
\parsep0.0ex\partopsep-0.3ex\itemsep2pt\topsep0.0ex}}{\end{list}}
% Dateinamen für interne Musterlösungen
\newcommand{\filename}[1]{%
\ifthenelse{\boolean{solutions}}{\framebox[50mm]{\parbox{40mm}{\textbf #1}}\vspace{6pt}}{}}
%   Taste
\newcommand{\taste}[1]{\small\textsf{#1}\normalsize}
%   geschützte Namen (NICHT in chapter, section usw. verwenden!)
\newcommand{\name}[1]{\textsl{#1}}
%   Fachbegriffe/Erklärung von Abkürzungen
\newcommand{\begriff}[1]{\index{#1}\textit{#1}}
%
\newcommand{\algorithmus}[2]{%
\vspace{4pt}\fboxsep 1mm \framebox[140mm]
{\parbox{135mm}{{\textbf{#1}}\vspace{2pt}#2}}\vspace{4pt}}
%
%   Dateinamen/Pfade
\newcommand{\datei}[1]{\texttt{#1}\normalsize}
%   Tip (in einer Box)
\newcommand{\tip}[1]{%
\vspace{4pt}\fboxsep 1mm \framebox[140mm]
{\parbox{135mm}{{\textbf{Tipp}}:\\ #1}}\vspace{4pt}}
%
%   Reflektion (in einer Box)
%
\newcommand{\reflection}[1]{
\begin{quote}\fboxsep 3mm\framebox[140mm][c]{\parbox{130mm}{{\textbf{Reflektion}:\\}#1}}\end{quote}}
%
%   Angabe, was die Studierenden lesen sollen (in einer Box)

\newcommand{\lesen}[1]{%
\begin{minipage}[c]{2.5cm}
\centering
\includegraphics[width=2cm]{\imagePath/Misc/buchicon}%
\end{minipage}%
\begin{minipage}[t]{12cm}%
#1%
\end{minipage}}
%
%   Angabe im Begleittext, was die Studierenden lesen sollen (in einer Box)
%   Dieses Icon wird für Angaben verwendet, die nicht verpflichtend zu lesen
%   sind. Also "nice-to-have", "wenn sie noch Zeit haben".
\newcommand{\vertiefen}[1]{%
\begin{minipage}[b]{2.5cm}
\centering
\includegraphics[width=2cm]{\imagePath/Misc/reading}%
\end{minipage}%
\begin{minipage}[b\newcommand{\firefox}{\texttt{Mozilla Firefox}}
\newcommand{\kde}{\texttt{KDE}}]{12cm}%
#1%
\end{minipage}}
%
% Jetzt kommt die Definition der Kontrollfrage; die Nummerierung
% für das ganze erhalten wir mit Hilfe von \item{\kontroll}.
%
% Ein Counter für die Kontrollfragen. Wir zählen diesen Counter selbst hoch
% mit einem entsprechenden Kommando, das wir in \item verwenden.
% 1
\newcounter{kontrollCounter}[section]
\newcommand{\kontroll}[0]{
    \refstepcounter{kontrollCounter}
    % Kapitelnummer.Zähler für das Item
    \arabic{chapter}.\arabic{kontrollCounter}
    % Das Label ist bis auf weiteres "kontrolle:Kapitelnummer:kontrollcounter
    \label{kontrolle:\arabic{chapter}:\arabic{kontrollCounter}}
}
% Jetzt kommt die Definition der Kontrollfrage; die Nummerierung
% für das ganze erhalten wir mit Hilfe von \item{\kontroll}.
\newcommand{\kontrollfrage}[1]{%
\begin{minipage}[c]{1.85cm}
\huge{\ding{46}}%
\end{minipage}%
\begin{minipage}[t]{12cm}%
\begin{itemize}#1%
\end{itemize}%
\end{minipage}}
%
% Einblenden von Musterlösungen
%
% Schalter für das ein- und ausblenden der Lösungen
\newboolean{solutions}
%
% Theorem-Umgebung für die Übungsaufgaben
% Wichtig: alle Attribute einstellen, dann die neue
% theorem-Umgebung mit newtheorem definieren!
% Oder, wie hier, durch das Einschließen in {}
%
{
% Zeilenumbruch bei Aufgaben-Überschrift
\theoremstyle{break}
% "Normaler" Font im Text
\theorembodyfont{\normalfont}
% Kapitelweise neu nummerieren
% 2
\newtheorem{auftitle}{Aufgabe}[section]
}
%
% Definition für die Kennzeichnung der Übungsaufgaben
%
\newcommand{\uebung}{%
\vspace*{11pt}%
\begin{tabular}{@{}p{2.25cm}@{}p{11.7cm}}%
\huge{\ding{45}}&\Large{\textbf{Übungsaufgaben}}%
\end{tabular}%
}
% Ein Counter für die Übungsaufgaben. Wir zählen diesen Counter selbst hoch
% mit einem entsprechenden Kommando, das wir in \item verwenden.
% 3
\newcounter{aufgabenCounter}[section]
\newcommand{\auf}[0]{
    \refstepcounter{aufgabenCounter}
    % Kapitelnummer.Zähler für das Item
    \arabic{chapter}.\arabic{aufgabenCounter}
}
% Der Text der Aufgaben steht im Ordner ../Uebungsaufgaben/aufgaben/aufgabenstellungen,
% so können wir die Dateien aus der Veranstaltung
% Wahrscheinlichkeitsrechnung und Statistik verwenden; und die neuen
% Aufgaben gehen in den allgemeinen Fundus ein.
%

% Umgebungen für Satz, Definition, Beweis, ... . Wir orientieren uns am Mathematik-Buch,
% dort gab es diese Umgebungen auch schon. Im Grunde sind das einfach
% wieder theorem-Umgebungen.
{
\setlength\theorempreskipamount{5pt plus 3pt minus 1.5pt}
\setlength\theorempostskipamount{5pt plus 1pt minus 1pt}
% "Normaler" Font im Text
\theorembodyfont{\normalfont}
% Die Namen sind so gewählt, dass sie kompatibel zu Beamer sind; dann können
% wir den Text aus Folien kopieren und umgekehrt auch.
% 4
\newtheorem{Satz}{Satz}[section]
\newtheorem{Fakt}{Fakt}[section]
\newtheorem{Definition}{Definition}[section]
}
%
{
\setlength\theorempreskipamount{8pt plus 3pt minus 1.5pt}
\setlength\theorempostskipamount{5pt plus 1pt minus 1pt}
\theoremstyle{break}
% "Normaler" Font im Text
\theorembodyfont{\normalfont}
\newtheorem{datensatz}{Datensatz}
}
% Text mit Pfad
\newcommand{\aufgabentext}[1]{\renewcommand{\labelenumi}{\alph{enumi})}\auftitle\label{#1}\input{../../uebungsaufgaben/Stochastik/aufgaben/aufgabenstellungen/#1}}
% Funktion für die Lösungs-Hinweise für eine Aufgabe. Der Text
% steht analog zu den Übungsaufgaben im Ordner aufgaben/loesungen.
\newcommand{\hinweistext}[1]{\renewcommand{\labelenumi}{\alph{enumi})}\input{../../Uebungsaufgaben/Stochastik/aufgaben/loesungen/#1}}
% Und jetzt die Funktionen, die wir im Text aufrufen
\newcommand{\aufgabe}[1]{\aufgabentext{#1}}
% Lösungshinweise im Anhang
\newcommand{\hinweis}[1]{\subsubsection*{Aufgabe \ref{#1}} \hinweistext{#1}}
%
% Datensatz-Texte und Daten in eigener Datei
\newcommand{\dataset}[1]{\begin{datensatz}\label{#1}\input{../../Uebungsaufgaben/Stochastik/datasets/#1}\end{datensatz}}
% Teilaufgaben alphabetisch nummerieren
\renewcommand{\labelenumi}{\arabic{enumi})}
%
% Marginalien
%
\newcommand{\randnotiz}[1]{\marginpar{\small{\textbf{#1}}}}
%
% Gabelschlüssel als Marginalie und Hinweis auf Praxisbezug
\newcommand{\praxisbezug}[0]{\randnotiz{\includegraphics[width=1cm]{\imagePath/Misc/gabel}}}
%
% alert aus beamer-Folien zu emph machen
%
\newcommand{\alert}[1]{\emph{#1}}
%
%
\newcommand{\titelseite}[1]{%
% Titelseite
\pagenumbering{roman}%
\thispagestyle{empty}%
\begin{titlepage}%
% Volle Zeilenbreite verwenden!
\centering%
\vspace*{2cm}%

\Huge{#1}\\\vspace*{0.5cm}%

\vspace*{10cm}%

\Large{\theAuthor{}}%

\Large{\theProject{}}%

\Large{\theSchool{}}%
\end{titlepage}}

% Titelseite mit zusätzlicher Grafik
%
% Das obligatorische Argument ist der Titel des Dokuments.
% Als Default wird das Logo der Stochastik-Veranstaltung als Titelbild
% verwendet. Mit Hilfe eines optionalen Arguments kann
% ein anderes Bild verwendet werden!
% Beispiel: \titelseite[\imagePath/misc/ameise]{Das Liebesleben der Ameisen}
% Aufruf mit Standardbild:
% \titelseite{Das Liebesleben der Ameisen}
%
\newcommand{\titelseiteMitBild}[2][\imagePath/logos/python]{%
%% Titelseite
\pagenumbering{roman}%
\thispagestyle{empty}%
\begin{titlepage}%
% Volle Zeilenbreite verwenden!
\centering%
\vspace*{0.5cm}%

\Huge{#2}\\\vspace*{0.5cm}%

\vspace*{3.0cm}%
\includegraphics[height=6cm]{#1}%

\vspace*{3.0cm}%
\Large{\theAuthor{}}%

\Large{\theProject{}}%

\Large{\theSchool{}}%
\end{titlepage}}
