Erzeugen Sie ein \lstinline!numpy!-Array mit den Werten
\lstinline![!%
\(-2\pi\), %
\(-\tfrac32\pi\), %
\(-\pi\), %
\(-\tfrac{\pi}2\), %
\(0\), %
\(\tfrac{\pi}2\), %
\(\pi\), %
\(\tfrac32\pi\), %
\(2\pi\)\lstinline!]!%
. F�hren Sie
die nachfolgenden Berechnungen darauf aus und geben Sie das jeweilige Ergebnis
mit \lstinline!print()! auf der Konsole aus.
\begin{enumerate}
	\item Berechnen Sie den Sinus, Kosinus und Tangens des Arrays.
	\item Multiplizieren Sie das Array elementweise mit sich selbst.
	\item Addieren Sie die Euler-sche Zahl $e$ zu jedem Wert im Array.
	\item Wandeln Sie das Array von Bogenma� in Gradma� um.
	\item Runden Sie s�mtliche Werte im Array auf zwei Nachkommastellen und
	      bilden Sie die Summe des Ergebnisses. Welchen Wert erwarten Sie f�r diese
				Summe? Unterscheidet sich das Ergebnis von der Summe der Werte im
				urspr�nglichen Array?
\end{enumerate}
