Um SymPy nutzen zu k�nnen muss zun�chst das Modul \lstinline!sympy! importiert werden.
\begin{lstlisting}
from sympy import *
\end{lstlisting}

Die Symbole k�nnen wie folgt erstellt werden.

\begin{lstlisting}
>>> x, y, z = symbols('x y z')
\end{lstlisting}


\begin{enumerate}
	\item Eine M�glichkeit die Ausdr�cke zu erstellen �ber einfache Zuweisung.
	Die Gleichheit wird mit der \lstinline!simplify!-Funktion und dem Vergleich mit \lstinline!0! gepr�ft.
\begin{lstlisting}[mathescape]
>>> expr1 = (x**2 + 1) * (x**2 - 1)
>>> expr2 = (x**2 - 1)
>>> simplify(expr1 - expr2) == 0
False
\end{lstlisting}
	\item Zum Substituieren wird die \lstinline!subs!-Methode verwendet.
	Dabei ist zu beachten dass dabei nicht der bestehende Ausdruck ge�ndert wird sondern ein Neuer erstellt wird.
\begin{lstlisting}
>>> expr3 = expr1.subs(x**2, x)
>>> simplify(expr3 - expr2) == 0
True
\end{lstlisting}
	\item Faktorisiert wird mit  \lstinline!factor!.
\begin{lstlisting}[mathescape]
>>> expr4 = factor(expr3)
>>> expr4
(x - 1)(x + 1)
\end{lstlisting}	
	\item Die in der Aufgabe davor durchgef�rte Akjtion ist das Faktorisieren.
	Um diese Aktion umzukehren muss der Term wieder Ausmultipliziert werden.
	Diese Operation kann durch das ausf�hren der \lstinline!simplify!-Funktion erreicht werden.
\begin{lstlisting}[mathescape]
>>> expr5 = simplify(expr4)
>>> expr5
$x^2 - 1$
\end{lstlisting}	
	

\end{enumerate}