Um NumPy nutzen zu k�nnen, muss zuvor das Modul \lstinline!numpy! importiert
werden.
\begin{lstlisting}
import numpy as np
\end{lstlisting}
Die Konstante \(\pi\) ist in NumPy als \lstinline!np.pi! verf�gbar. Damit kann
das NumPy-Array wie folgt definiert werden:
\begin{lstlisting}
a = np.array([-2*np.pi, -1.5*np.pi, -1*np.pi, -0.5*np.pi, 0,
    0.5*np.pi, 1*np.pi, 1.5*np.pi, 2*np.pi])
\end{lstlisting}

\begin{enumerate}
	\item Die ben�tigten trigonometrischen Funktionen hei�en \lstinline!np.sin()!,
	      \lstinline!np.cos()! und \lstinline!np.tan()!.
	\item Um zwei Arrays miteinander elementweise zu Multiplizieren, wird der
	      \lstinline!*!-Operator benutzt.
	\item Die Euler-sche Zahl ist als \lstinline!np.e! in NumPy enthalten. Um eine
	      Zahl zu jedem Wert eines Arrays zu addieren, wird der \lstinline!+!-%
				Operator genutzt.
	\item Die Funktion hierzu hei�t \lstinline!np.degrees()!.
	\item Die Rundungsfunktionen hei�t \lstinline!np.around()!. Der erste
				Parameter ist ein Array, der Zweite ist die Anzahl der Stellen, auf die
				gerundet wird.
\end{enumerate}
