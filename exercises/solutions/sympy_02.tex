Die Funktionen k�nnen wir folgt erzeugt werden:
\begin{lstlisting}
>>> x, y, z = symbols('x y z')
>>> f = x**5 + y**2 + 3
>>> g = x**3 * y**2 * z + 4
>>> h = x**-3
\end{lstlisting}

\begin{enumerate}
	\item Die dritte Ableitung nach x von f kann wie folgt berechnet werden:
\begin{lstlisting}[mathescape]
>>> diff(f, x, 3)
$60x^2$
\end{lstlisting}
	\item 
\begin{lstlisting}[mathescape]
>>> diff(g, x, 3, y, 2, z)
12
\end{lstlisting}
	\item Integriert wird mit \lstinline!integrate!.
\begin{lstlisting}[mathescape]
>>> integrate(f, (x, 0, 6))
$6y^2+7794$
\end{lstlisting}
	\item Der Grenzwert wird mit \lstinline!limit! berechnet.
\begin{lstlisting}[mathescape]
>>> limit(h, x, 0, '+')
$\infty$
>>> limit(h, x, 0, '-')
$-\infty$
\end{lstlisting}
	\item Um des Gleichungssystem zu l�sen empfiehlt es sich zun�chst alle Gleichungen so umzuformen dass immer eine $0$ auf der rechten Seite steht.
	Gel�st wird das Gleichungssystem dann mit der \lstinline!linsolve!-Funktion.
	Eine M�glichkeit sieht wie folgt aus:

\begin{lstlisting}[mathescape]
>>> eq1 = -x + y + z
>>> eq2 = x - 3*y - 2*z - 5
>>> eq3 = 5*x + y + 4*z - 3
>>> linsolve([eq1, eq2, eq3], (x, y, z))
{(-1, -4, 3)}
\end{lstlisting}

\end{enumerate}