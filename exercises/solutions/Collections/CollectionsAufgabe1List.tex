\begin{enumerate}
	\item Um das gew�nschte Resultat zu erhalten, ben�tigen wir eine Schleife. Diese f�ngt bei dem Wert 1 an.
	Mithilfe der \lstinline$append()$-Methode k�nnen wir die Elemente unserer List hinzuf�gen.
	\lstinputlisting[language=Python, firstline=1,lastline=8]{exercises/src/Collections/CollectionsAufgabe1List.py}

	\item Die L�nge der List auf der Konsole ausgeben.
	\lstinputlisting[language=Python, firstline=12,lastline=14]{exercises/src/Collections/CollectionsAufgabe1List.py}

	\item Den Wert 4 aus der List l�schen.
	\lstinputlisting[language=Python, firstline=17,lastline=19]{exercises/src/Collections/CollectionsAufgabe1List.py}
	
	\item Das Einf�gen der 22 an der Stelle zwischen den Werten 3 und 5.
	\lstinputlisting[language=Python, firstline=22,lastline=24]{exercises/src/Collections/CollectionsAufgabe1List.py}

	\item Den Wert 1 am Ende der List hinzuf�gen.
	Dabei sollte Ihnen aufgefallen sein, das der Wert als Duplikat mitaufgenommen wurde.
	\lstinputlisting[language=Python, firstline=27,lastline=29]{exercises/src/Collections/CollectionsAufgabe1List.py}
	
	\item Sortieren der List.
	\lstinputlisting[language=Python, firstline=32,lastline=34]{exercises/src/Collections/CollectionsAufgabe1List.py}
\end{enumerate}