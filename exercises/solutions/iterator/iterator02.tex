Zuerst muss ein Generator definiert werden, wie im zugeh�rigen Kapitel beschrieben.
Durch eine range()-Anweisung wird eine Liste erzeugt, die Nummern von len(data)-1 (L�nger von Data - 1) bis -1 enth�lt mit der Schrittweite -1.
Die erzeugte Liste z�hlt mit jedem Element eins runter, bis sie bei -1 ankommt. 
Durch eine for-Schleife wird die Liste durchlaufen und bei jedem Durchlauf mit \lstinline$yield$ ein Element von Data zur�ckgegeben - jedoch in umgekehrter Reihenfolge.

\lstinputlisting{exercises/src/iterator/iterator02.py}