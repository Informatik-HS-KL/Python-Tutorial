% !TeX root = ../../pythonTutorial.tex

\section{Datenbanken}\label{database:einleitung}
In diesem Kapitel wird die Benutzung zweier verschiedener Datenbanksysteme, in der Programmiersprache Python, demonstriert. 
\subsection{Relationale Datenbanken}\label{database:relationaldatabase}
Relationale Datenbanken, die bereits im dritten Semester in der Vorlesung \glqq Datenbanken\grqq  durchgenommen wurden, dienen zur elektronischen Speicherung von Daten. In folgendem Kapitel wird das Einbinden von SQLite in eine Python-Anwendung genauer erl�utert.
\subsubsection{SQLite}\label{database:sqlite}
SQLite ist ein relationales Datenbankmanagementsystem, welches in einer C-Bibliothek enthalten ist. Anders als beispielsweise MySQL nutzt  SQLite nicht das client-server model, sondern ist sp�ter im fertigen Programm lokal integriert. Eingesetzt wird SQLite sehr h�ufig in Android-Apps. Diese speichern Nutzerdaten lokal auf dem Ger�t. Im weiteren Verlauf des Kapitels wird das Einbinden von SQLite in eine Python-Anwendung erl�utert.

\textbf{Einbinden von SQLite}

Um SQL in einem Python-Projekt verwenden zu k�nnen, wird lediglich ein import statement ben�tigt.
\begin{lstlisting}
#import sqlite
import sqlite3
\end{lstlisting}\label{database:lst:importsqlite}
Anschlie�end muss eine neue Datenbank angelegt werden. Hierf�r kann die Methode connect() verwendet werden. F�r diese Methode muss der �bergabeparameter aus einem string mit der Endung \glqq.db\grqq  �bergeben werden.
\begin{lstlisting}
#connect sqlite
connection = sqlite3.connect("beispiel.db")
\end{lstlisting}\label{database:lst:connect}
Falls alle Tabellen angelegt und wie gewollt bef�llt wurden, k�nnen mit der Methode commit() alle �nderungen an der Datenbank abgespeichert werden.
\begin{lstlisting}
#commit
connection.commit()
\end{lstlisting}\label{database:lst:commit}
Um die Verbindung zur Datenbank zu beenden benutzt man die Methode close()
\begin{lstlisting}
#close
connection.close()
\end{lstlisting}\label{database:lst:close}

\textbf{Create bei SQLite}

In Folge dessen k�nnen nun einzelne Tabellen zur Datenbank hinzugef�gt werden. Dies kann umgesetzt werden, indem ein neuer \glqq sql\_command\grqq angelegt wird, welcher ein korrektes SQL-Kommando beinhalten muss.

\begin{lstlisting}
#create
sql_command = """
CREATE TABLE mitarbeiter( 
mitarbeiterid INTEGER PRIMARY KEY, 
vname VARCHAR(20), 
nname VARCHAR(30), 
geschlecht CHAR(1), 
beitritt DATE,
geburtstag DATE);"""
\end{lstlisting}\label{database:lst:create}
 
\textbf{Insert bei SQLite}

Eine Tabelle wird im Anschluss wie folgt bef�llt.
\begin{lstlisting}
#insert
sql_command = """INSERT INTO mitarbeiter
(mitarbeiterid, vname, nname, geschlecht, geburtstag)
VALUES (NULL, "Peter", "Maffay", "m", "30.08.1949");"""
\end{lstlisting}\label{database:lst:insert}

\textbf{Delete bei SQLite}

Eine Tabelle oder einen bestimmten Mitarbeiter kann durch einen \glqq delete\grqq  Befehl wieder entfernt werden. In folgendem Listing werden zwei M�glichkeiten, Daten aus der Datenbank zu entfernen, aufgezeigt.
\begin{lstlisting}
#delete
sql_command1 = """DELETE FROM mitarbeiter 
WHERE mitarbeiterid = 1;???
sql_command2 = """DELETE FROM mitarbeiter;???
\end{lstlisting}\label{database:lst:delete}

\textbf{Update bei SQLite}

Um einen Eintrag im Nachhinein zu �ndern, kann durch den \glqq Update\grqq Befehl ein oder mehrere bestimmte Eintr�ge ge�ndert werden. Im Listing wird der Vorname des Mitarbeiters mit der MitarbeiterId 1 auf Peter gesetzt.
\begin{lstlisting}
#update
sql_command = """UPDATE mitarbeiter SET vname=?Peter?
WHERE mitarbeiterid = 1;"""
\end{lstlisting}\label{database:lst:update}

\textbf{Select bei SQLite}

M�chte man Eibtr�ge der Datenbank auslesen benutzt man den Select-Befehl.
Dieser Erm�glicht es uns einen oder mehrere Beitr�ge auszulesen.
\begin{lstlisting}
#select
sql_command1 = """SELECT * FROM mitarbeiter;"""
sql_command2 = """SELECT * FROM mitarbeiter 
WHERE mitarbeiterid = 1;"""

\end{lstlisting}\label{database:lst:select}

\subsection{NoSQL Datenbanken}

NoSql steht f�r \glqq not only SQL\grqq  Hier wird allerdings SQL als Synonym f�r relationale Datenbanksysteme verwendet. Die Grundidee ist die, dass man nicht von Anfang an durch alte Gewohnheit direkt eine relationale Datenbank benutzt. Sondern sich f�r ein Datenbanksystem entscheidet, dass am Besten zum geplanten Projekt passt. Entstanden sind solche NoSQL Datenbanken unter anderem durch soziale Netzwerke. Hier m�ssen mehrere Millionen Daten sehr schnell gespeichert und abgerufen werden. Eine solche Masse an Anfragen stellte ganz neue Anforderung an Datenbanksysteme.
Die wichtigsten Kategorien von NoSQL-Datenbanken sind Key-Value, spaltenorientierte, Value und dokumentenorientierte Datenbanken. In folgendem Kapitel, wird das Einbinden der dokumentenbasierten NoSQL Datenbank MongoDB genauer beschrieben.

\subsubsection{MongoDB}
MongoDB ist eine open-source dokumentenbasierte NoSQL-Datenbank, die eine hohe Performance und automatische Skalierung bietet. Als \glqq record\grqq
bezeichnet man in MongoDB eine Datenstruktur (key/value) mit field und den dazugeh�rigen values. Diese MongoDB Dokumenten sind �hnlich zu den uns bekannten JSON-Objekts. In diesem Kapitel wird die Einbindung und die Benutzung einer MongoDB Datenbank genauer erl�utert.

\textbf{Einbinden von MongoDB}

Um eine Verbindung mit MongoDB herzustellen muss zun�chst die Python distribution PyMongo installiert werden.
\begin{lstlisting}
#import mongodb
import pymongo
\end{lstlisting}\label{database:lst:importmongodb}
Danach muss mit mongod eine MongoDB Instanz gestartet werden.
Als n�chstes muss ein MongoClient erstellt werden, welcher auf die laufende mongod Instanz zugreift und dabei mit dem default Host und default Port verbunden.
\begin{lstlisting}
#create client
from pymongo import MongoClient
client = MongoClient()
\end{lstlisting}\label{database:lst:client}
Man kann den Host und Port aber auch explizit spezifizeren, indem man eines der folgen Formate benutzt.
\begin{lstlisting}
#Host Name und Passwort spezifizieren
client1 = MongoClient('localhost', 12345)
client2 = MongoClient('mongodb://localhost:12345/')
\end{lstlisting}\label{database:lst:host}
In PyMongo greift man mit attribute style access auf die Datenbanken zu, da eine Instanz von MongoDB mehrere unabh�ngige Datenbanken unterst�tzt. Dabei kann man die beiden folgenden Formate benutzen.
\begin{lstlisting}
#Zugriff auf die Datenbank in zwei Varianten
db1 = client.test_database
db2 = client['test-database']
\end{lstlisting}\label{database:lst:databaseconnection}
Das Equivalent zu Tabellen in relationalen Datenbanken in MongoDB wird Connections genannt. Diese bestehen aus mehreren Dokumenten. Auf diese greift man genauso zu wie auf die Datenbank.
\begin{lstlisting}
#Zugriff auf Connections in zwei Varianten
collection = db.test_collection
collection = db['test-collection']
\end{lstlisting}\label{database:lst:tableconnection}
Die oben genannten Dokumente (JSON-Style) sind die Repr�sentanten und Speicher der Daten in der Datenbank.
\begin{lstlisting}
#JSON
import datetime
test = {	"author": "Lukas",
"text": "Hallo",
"tags": ["hallo", "pymongo"]
?date?: datetime.datetime.utcnow()}
\end{lstlisting}\label{database:lst:json}

\textbf{Insert und Delete bei MongoDB}

Um ein Dokument einzuf�gen, muss man die insert\_one() Methode benutzen.
\begin{lstlisting}
#insert_one
test = db.test
test_id = test.insert_one(test).inserted_id
test_id
ObjectId('...')(Ausgabe)
\end{lstlisting}\label{database:lst:insertone}

Beim Einf�gen eines Dokuments wird diesem automatisch eine \_id zugewiesen, falls es noch keine vorher bestimmte \_id hat. Diese \_id muss einzigartig in der Collection sein. Nachdem man nun dieses Dokument hinzugef�gt hat, wurde eine Test Collection erstellt. Indem man nun eine Liste aller Collections anzeigen l�sst, kann man dies best�tigen.
\begin{lstlisting}
#collection_names
db.collection_names(include_system_collections=False)
\end{lstlisting}\label{database:lst:collectionnames}
Um ein bereits hinzugef�gtes Dokument zu l�schen wird die Methode delete\_one() genutzt.

\textbf{Select bei MongoDB}

Man kann mit der folgenden Methode auf bestimmte oder das erste (kein Parameter) Dokument einer Collection zugreifen.
\begin{lstlisting}
#find_one mit autor
test.find_one({?author?: ?Lukas?})
\end{lstlisting}\label{database:lst:findone}
Mithilfe der \_id kann man auch auf die einzelnen Dokumente zugreifen.
\begin{lstlisting}
#find_one mit id
test.find_one({?_id?: test_id})
\end{lstlisting}\label{database:lst:findone}

\textbf{Update bei MongoDB}

Um einem Dokument neue Parameter zu geben, wird die Funktion update\_one(altes Dokument, neues Dokument) genutzt. Das neue Dokument muss den Operator set
enthalten. 
\begin{lstlisting}
#update
newvalue { "$set": { "author": "Lukas" } }
\end{lstlisting}\label{database:lst:update}
Die Methoden insert\_one() und find\_one() k�nnen in abge�nderter Form auch f�r mehrere Dokumente genutzt werden.
insert\_many() benutzt man um mehrere Dokumente hinzuzuf�gen

Um auf mehrere Dokumente zuzugreifen, benutzt man find().
Die Anzahl der Dokumente wird in unserem Fall mit test.count\_documents({}).
Mit .sort kann man die Ergebnisse nach verschiedenen Parametern sortieren.
Nun wird ein neuer Index erstellt.
\begin{lstlisting}
#XXXXXXXX
result = db.profiles.create_index(
[('user_id',pymongo.ASCENDING)],unique=True)
\end{lstlisting}\label{database:lst:XXXXXX}
Das hat zur Folge, dass es nun eine \_id und eine user\_id gibt, welche einzigartig sein m�ssen.
Zuletzt wird die Verbindung mit der Datenbank unterbrochen.
\begin{lstlisting}
#close
client.close
\end{lstlisting}\label{database:lst:close}
