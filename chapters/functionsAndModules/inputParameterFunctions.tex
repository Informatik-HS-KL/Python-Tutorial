% !TeX root = ../../pythonTutorial.tex
\section{Input-Parameter}
\label{inputParameterFunctions}
In diesem Abschnitt sehen wir uns die verschiedenen Arten von �bergabeparametern genauer an, sowie die M�glichkeiten, diese an eine Funktion zu �bergeben.

\begin{lstlisting}[language=Python, label=inputParameterFunctions:lst:primitiveParameter]
# �bergabe eines primitiven Datentyps

def myFunction(x):
    x = 1

x = 2
myFunction(x)
print(x)

# Ausgabe:
# 2
\end{lstlisting}

Dieser Code 
%(\ref{inputParameterFunctions:lst:primitiveParameter}) 
f�hrt dazu, dass der Wert '2' ausgedruckt wird. 
Aus dem vorherigen Abschnitt �ber G�ltigkeitsbereiche wissen wir, dass es sich hierbei - trotz gleichem Namen - --um zwei verschiedene Variablen handelt. 
Eine Integer-Variable im lokalen und eine im globalen G�ltigkeitsbereich. Dies gilt jedoch nur f�r primitive Datentypen, f�r nicht-primitive Datentypen verh�lt es sich anders:

\begin{lstlisting}[language=Python, label=inputParameterFunctions:lst:nonPrimitiveParameter]
# �bergabe eines nicht-primitiven Datentyps

def myFunction(x)
    x[0] = 100

x = [0,1,2]
myFunction(x)
print(x)
\end{lstlisting}

Da es sich hier 
%(\ref{inputParameterFunctions:lst:nonPrimitiveParameter}) 
nicht um einen primitiven Datentyp handelt, stellt der Input-Parameter eine Referenz dar, womit der Wert von x auch in der Methode ge�ndert wird. Beim Ausgeben erhalten wir: \lstinline$[100,1,2]$.

Es ist wichtig, sich dieses Verhalten klar zu machen, da beim objekt-orientierten Programmieren sonst Fehler auftreten.

\kontrollfrage{
	\item[\kontroll] Unter welcher Voraussetzung f�hren �nderungen eines �bergabeparameters innerhalb einer Funktion zu �nderungen au�erhalb?
	\item[\kontroll] Welche nicht-primitiven Datentypen kennen wir?
}

\subsection{Arten von Input-Parametern}
\label{inputParameterFunctions:subsection:inputParameter}
In folgendem Unterabschnitt gehen wir auf verschiedene Arten von Input-Parametern ein.


Diese Art von Parametern d�rfte bereits bekannt sein.\randnotiz{Positional} Es handelt sich um eine endliche Anzahl von Parametern, die man von links nach rechts liest.

\begin{lstlisting}[language=Python, label=inputParameterFunctions:lst:positionalParameter]
# Funktion mit positionalen Parametern

def myFunction(x,y,z):
    ...
    myFunction(1,2,3)
\end{lstlisting}


Bei \randnotiz{Schl�sselwort} Schl�sselwort-Parametern ist die Reihenfolge der Parameter nicht elementar, da die Variablen �ber den Namens-Wert �bergeben werden.

\warning{Bei der Verwendung von positionalen und Schl�sselwort-Parametern, m�ssen positionale Parameter immer links der Schl�sselwort-Parameter stehen.}

\begin{lstlisting}[language=Python, label=inputParameterFunctions:lst:keywordParameter]
# Funktion mit Schl�sselwort-Parametern

def myFunction(x,y,z):
    ...
    myFunction(x = 1, z = 3, y = 2)
\end{lstlisting}

Standard-Parameter \randnotiz{Standard} folgen einer �hnlichen Syntax wie Schl�ssel"-wort-Para"-meter, sie werden jedoch in der Funktionsdefinition festgelegt und nicht beim Aufruf.
Zu beachten ist, dass beim Verwenden von positionalen und Standard-Parametern alle Standard-Parameter \emph{nach} den positionalen stehen m�ssen.

\begin{lstlisting}[language=Python, label=inputParameterFunctions:lst:standardParameter]
# Funktion mit Standard-Parametern

def myFunction(x, y = 10, z = 20):
    print(x, y, z)

myFunction(0)                       # 0 10 20
myFunction(y = 20, x = 10, z = 30)  # 10 20 30
myFunction(0, z = 1)                # 0 10 1
\end{lstlisting}

\label{inputParameterFunctions:subsection:multipleParameter}
Variable Parameter \randnotiz{Variable} erm�glichen quasi eine unendliche Anzahl an �bergabe-Parametern.
M�chte ein Nutzer beispielsweise den Durchschnitt mehrerer Zahlen wissen, so k�nnte die
Funktion wie folgt aussehen:

\begin{lstlisting}[language=Python, label=inputParameterFunctions:lst:restParameter]
# Funktion mit variablen Parametern

def average(*numbers):
    value = 1
    for number in numbers:
        value = value * number

    value = value / len(numbers)
    # durch die Anzahl an Parametern
    return value

result = average(5, 5, 10, 8, 2)
print(result) # druckt das Ergebnis 6
\end{lstlisting}

Die multiplen Parameter werden als Liste gehandhabt.
Bei der Definition von variablen Parametern in einer Funktion ist es wichtig, dass sie als letztes Argument stehen.

\kontrollfrage{
	\item[\kontroll] Worin liegt der Unterschied zwischen Standard- und Schl�sselwort-Parametern?
	\item[\kontroll] Worauf ist bei gleichzeitiger Nutzung von positionalen und Standard-Parametern zu achten?
	\item[\kontroll] Welchen Nutzen bieten variable Parameter? Gibt es Kontrollstrukturen, bei denen sich variable Parameter anbieten?
}
