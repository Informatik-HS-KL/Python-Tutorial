% !TeX root = ../../pythonTutorial.tex
\subsection{Parallelit�t in Python}
\label{parallelit�t_in_python:subsection:parallelit�t_in_python}

In der Referenzimplementierung CPython des Python-Interpreters existiert ein Konstrukt,
welches eine echte parallele Ausf�hrung von Python-Code verhindert.
Bei diesem Konstrukt handelt es sich um das sogenannte Global-Interpreter-Lock, oder kurz GIL.
F�r die Verwendung eine GILs in Python sprechen mehrere Punkte.
Python wurde so entworfen, dass es leicht zu verwenden ist, um den
Entwicklungsprozess zu beschleunigen.
Ein GIL verhindert, dass sich mehrere Threads gleichzeitig in der Ausf�hrung befinden k�nnen,
was die Entwicklung von Multithreaded-Programmen erheblich erleichtert.
Weiterhin wurde der Funktionsumfang von Python durch viele in C geschriebene Erweiterungen erg�nzt.
Um Inkonsistenzen zu verhindern, ben�tigen diese C-Erweiterungen eine threadsichere
Speicherverwaltung, die durch das GIL garantiert ist.
Die Verwendung des GILs erleichtert auch die Integration von nicht threadsicheren C-Bibliotheken.
Da das Einbinden von C-Bibliotheken durch das GIL leicht zu realisieren ist, existieren
viele Erweiterungen zu Python, die zur weiten Verbreitung von Python f�hrten.

Durch die Verwendung des GILs ist eine parallele Programmierung in Python
allerdings nicht g�nzlich ausgeschlossen.
Lediglich Aufgaben, die CPU-gebunden sind, sind hierdurch betroffen.
I/O-gebundene Aufgaben, wie zum Beispiel die Anfragea von Daten aus einer Datenbank, oder
die Abfrage von Benutzereingaben, k�nnen auch trotz des GILs parallel ausgef�hrt werden.
Durch die Verwendung von mehreren Prozessen ist es auch m�glich, parallele
Ausf�hrung von Python-Code zu erreichen.
Dies funktioniert, da jeder Python-Prozess seinen eignen Python-Interpreter und somit
auch ein eigenes GIL besitzt.