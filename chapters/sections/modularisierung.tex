% !TeX root = ../../pythonTutorial.tex

\section{Modularisierung}
Unter Modularisierung versteht man die Aufteilung des Programmcodes in Modulen. Ein Modul stellt dabei verschiedene Datentypen, Funktionen und Werte bereit. Durch das Auslagern dieser Elementen in Modulen, wird das Projekt besser strukturiert. Zudem wird das Hauptprogramm aufger�umt, was zu die Leserlichkeit des Quellcodes erh�ht.

Bei den Modulen wird zwischen lokalen und globalen Modulen unterschieden. W�hrend lokale Module nur in einem Projekt genutzt werden, k�nnen globale Module projekt�bergreifend verwendet werden. Ein Beispiel f�r ein globales Modul ist das Modul \texttt{math} der Standartbibliothek, welches mathematische Funktionen und Konstanten zur bereitstellt.

Ob ein Modul lokal oder global ist, h�ngt vom Speicherort ab. Ein lokales Modul ist in der Regel im Programmverzeichnis oder in einem Unterverzeichnis davon hinterlegt. Ein globales Modul wird in einem bestimmten Verzeichnis der Python-Installation angelegt. Hierzu geh�rt beispielsweise das Verzeichnis \textit{site-packages}, indem auch einige Module von Drittanbietern installiert werden.

\subsection{Erstellung eines lokalen Moduls}

Zur Erstellung eines lokalen Moduls wird eine Datei \textit{meinModul.py} im Programmverzeichnis angelegt:
\begin{lstlisting}
def hallo(fname):
	print("Hallo " + fname)
	
NAME = "Tobias"
\end{lstlisting}
Der Inhalt dieser Datei  f�hrt keinerlei Code aus, sondern stellt lediglich eine Funktion und eine Konstante bereit, welche sp�ter im Hauptprogramm genutzt werden k�nnen.
