% !TeX root = ../../pythonTutorial.tex

\section{Modularisierung}
Unter Modularisierung versteht man die Aufteilung des Programmcodes in Modulen. Ein Modul stellt dabei verschiedene Datentypen, Funktionen und Werte bereit. Durch das Auslagern dieser Elementen in Modulen, wird das Projekt besser strukturiert. Zudem wird das Hauptprogramm aufger�umt und die Leserlichkeit des Quellcodes erh�ht.

Bei den Modulen wird zwischen lokalen und globalen Modulen unterschieden. W�hrend lokale Module nur in einem Projekt genutzt werden, k�nnen globale Module projekt�bergreifend verwendet werden. Ein Beispiel f�r ein globales Modul ist das Modul \texttt{math} der Standartbibliothek, welches mathematische Funktionen und Konstanten bereitstellt.

Ob ein Modul lokal oder global ist, h�ngt vom Speicherort ab. Ein lokales Modul ist in der Regel im Programmverzeichnis oder in einem Unterverzeichnis davon hinterlegt. Ein globales Modul wird in einem bestimmten Verzeichnis der Python-Installation angelegt. Hierzu geh�rt beispielsweise das Verzeichnis \textit{site-packages}, indem auch einige Module von Drittanbietern installiert werden.

\subsection{Erstellung eines lokalen Moduls}
Zur Erstellung eines lokalen Moduls wird eine Datei \textit{meinModul.py} im Programmverzeichnis angelegt:
\begin{lstlisting}
def hallo(fname):
	print("Hallo " + fname)
    
MAX = "Maximilian"
\end{lstlisting}
Der Inhalt dieser Datei  f�hrt keinerlei Code aus, sondern stellt lediglich eine Funktion und eine Konstante bereit, welche sp�ter im Hauptprogramm genutzt werden k�nnen.

\subsection{Module verwenden}
Um lokale und globale Module verwenden zu k�nnen, m�ssen diese zun�chst im Hauptprogramm eingebunden werden. Hierzu wird die \texttt{import}-Anweisung verwendet. Prinzipiell ist es egal wo dies im Quellcode geschieht. Es empfiehlt sich jedoch alle Module zu Beginn des Quelltextes einzubinden. Die \texttt{import}-Anweisung besteht aus dem dem Schl�sselwort \texttt{import} gefolgt vom Namen des Moduls. Im Folgenden wird das zuvor erstellte Modul eingebunden:
\begin{lstlisting}
import meinModul
\end{lstlisting}

Zu Beachten ist, dass beim Einbinden von Modulen die Dateiendung entf�llt. Mit einer \texttt{import}-Anweisung k�nnen auch mehrere Module eingebunden werden. Hierzu werden die Modulnamen hinter dem Schl�sselwort, durch Kommas getrennt, aufgelistet:
\begin{lstlisting}
import meinModul, math
\end{lstlisting}
anstelle von:
\begin{lstlisting}
import meinModul
import math
\end{lstlisting}

Durch das Einbinden eines Moduls, wird ein neuer Namensraum mit dem Modulnamen erzeugt. �ber den Namensraum k�nnen alle Elemente des Moduls angesprochen werden:
\begin{lstlisting}
import meinModul
meinModul.hallo(Welt)
meinModul.hallo(MAX)
\end{lstlisting}
\newpage
Ausgabe:
\begin{lstlisting}
Hallo Welt
Hallo Maximilian
\end{lstlisting}

Mithilfe der \texttt{import/as}-Anweisung ist es m�glich den Namen des Namensraum zu ver�ndern:
\begin{lstlisting}
import meinModul as myBib
myBib.hallo(Welt)
\end{lstlisting}

\textbf{Hinweis:}\\
Es gilt zu beachten, dass nach dieser Anweisung das Modul \textit{meinModul} ausschlie�lich �ber den Namensraum \textit{myBib} zu erreichen ist.

Python bietet zudem die M�glichkeit \textit{einzelne} Elemente aus einem Modul zu importieren. Hierzu wird die \texttt{from/import}-Anweisung verwendet:
\begin{lstlisting}
from math import sin, cos, tan, sinh, cosh, tanh
\end{lstlisting}

Falls die Liste der zu importierenden Elemente etwas L�nger ausfallen sollte, kann die Aufz�hlung in mehreren Zeilen erfolgen, um so die Leserlichkeit des Codes zu verbessern. Zur Realisierung m�ssen zun�chst die einzubindenden Elemente in runden Klammern gesetzt werden. Anschlie�end k�nnen beliebig viele Zeilenumbr�che innerhalb der runden Klammern erfolgen. Folgendes Beispiel zeigt eine \texttt{from/import}-Anweisung mit einem Zeilenumbruch:
\begin{lstlisting}
from math import (sin, cos, tan,
                    sinh, cosh, tanh)
\end{lstlisting}

Mit einem Stern k�nnen \textit{alle} Elemente des Moduls imporiert werden:
\begin{lstlisting}
from meinModul import *
\end{lstlisting}

Es gilt zu beachten, dass bei der \texttt{from/import}-Anweisung kein eigener Namensraum erzeugt wird. Die Elemente werden stattdessen in den \textit{globalen} Namensraum eingebunden. Dies hat zur Folge, dass der Namensraum bei einem Zugriff auf ein Element nicht mehr angegeben wird:
\begin{lstlisting}
from meinModul import hallo
hallo(Welt)		# statt: meinModul.hallo(Welt)
\end{lstlisting}

\textbf{Achtung:}\\
Da die Elemente bei der \texttt{from/import}-Anweisung in den globalen Namensraum eingebunden werden, k�nnen bereits vorhandene Referenzen kommentarlos �berschrieben werden. Hierzu ein kleines Beispiel:
\begin{lstlisting}
pi = 42
print("pi hat den Wert:")
print(pi)

from math import pi
print("pi hat den Wert:")
print(pi)
\end{lstlisting}

Ausgabe:
\begin{lstlisting}
pi hat den Wert:
42
pi hat den Wert:
3.141592653589793
\end{lstlisting}

Um derartige Fehler zu minimieren sollte die \texttt{from/import}-Anweisung sparsam verwendet werden - also nur dann,wenn einzelne Elemente aus einem Modul ben�tigt werden. Wenn alle oder fast alle Elemente aus einem Modul ben�tigt werden, empfiehlt es sich die \texttt{import}-Anweisung zu verwenden.
