\section{Tkinter}

Tk ist ein freies, plattform�bergreifendes GUI-Toolkit von Scriptics
(fr�her von Sun Labs entwickelt) zur Programmierung von grafischen
Benutzeroberfl�chen (GUIs). Urspr�nglich f�r die Programmiersprache Tcl
(Tool command language) entwickelt, existieren heute Anbindung an diverse
Programmiersprachen.
Unter vielen dynamischen Sprachen ist Tk das Standard-GUI und kann umfangreiche,
ab dem Release 8.0 mit nativem Look-and-Feel versehene Anwendungen erstellen,
die unver�ndert unter Windows, Mac OS X und Linux laufen.

Tkinter ist eine Sprachanbindung f�r das am h�ufigsten verwendete GUI-Toolkit
Tk f�r die Programmiersprache Python.
Der Name steht als Abk�rzung f�r Tk interface. Tkinter war das erste GUI-Toolkit
f�r Python, weshalb es inzwischen auf Mac OS X und Windows zum Lieferumfang
von Python geh�rt.

Tkinter besteht aus einer Reihe von Modulen. Das Tk interface wird von einem
bin�ren Erweiterungsmodul namens \lstinline$_tkinter$ bereitgestellt. Dieses Modul enth�lt
die Low-Level-Schnittstelle zu Tk und sollte niemals direkt von
Anwendungsprogrammierern verwendet werden. Es handelt sich in der Regel um eine
Shared Library (oder DLL), kann aber in einigen F�llen statisch mit dem
Python-Interpreter verkn�pft sein \cite{tkinter}.

\section{Einbindung}

Wir beginnen mit dem Import des Tkinter-Moduls, welches alle Klassen und Funktionen enth�llt, die
f�r die Arbeit mit dem Tk-Toolkit erforderlich sind.

\lstinputlisting[language=Python]{chapters/sections/GUI/GUI_Imports.py}
\label{gui:gui_imports}

Ab Python 3.X.X wird Tkinter nach der Import-Anweisung klein geschrieben.

\section{Hallo Welt in grafischer Benutzeroberfl�che}

Um Tkinter zu initialisieren, muss ein Tk Wurzel-Element erstellt werden.
Dies ist ein gew�hnliches Fenster, mit einer Titelleiste und anderen Dekorationen,
die von dem Betriebssystem bereitgestellt werden.
Generell sollte nur ein Wurzel-Element f�r jedes Programm erstellt werden, welches auf h�chster
Ebene vor allen anderen Widget-Elementen steht.

Im folgenden Beispiel wird dazu ein Wurzel-Element mit dem Namen \lstinline$root$ initialisiert.

\lstinputlisting[language=Python]{chapters/sections/GUI/GUI_HelloWorld.py}
\label{gui:gui_helloworld}

Damit das Fenster mit Inhalt gef�llt werden kann, muss dem Wurzel-Element weitere
Widgets als Kinder hinzugef�gt werden. Dazu wurde ein Label mit dem Namen \lstinline$label_1$ angelegt.
Die Parameterliste des \lstinline$Label$-Widgets nimmt als erstes Argument den Namen des Eltern- bzw. Wurzel-Element und
als zweites Argument den Content bzw. Inhalt des Widgets entgegen.

Als n�chstes ben�tigt Tkinter die Anweisung, wie das der Inhalt das Fenster angezeigt werden soll.
Dazu muss eine der Layout-Methoden, in diesem Fall \lstinline$label_1.grid()$ aufgerufen werden.
Das Fenster wird jedoch erst sichtbar, sobald die Ereignisschleife \lstinline$mainloop()$ betreten wird.

Das Programm wird nun solange ausgef�hrt und bleibt in der Ereignisschleife, bis das Fenster geschlossen wird.
In der Mainloop-Ereignisschleife werden nicht nur Events wie Nutzereingaben (z.B. Mausklicks oder Tastatureingaben)
verarbeitet, sondern auch die des Fenstersystems (z.B. Redraw-Ereignisse und Fenster-Konfigurationsmeldungen) und
Ereignisse von Tkinter selbst.

\section{Layout-Manager}

\section{Widgets}

\subsection{Label}
\subsection{Entry}
\subsection{Button}
\subsection{Frame}

\section{Pop-ups}
