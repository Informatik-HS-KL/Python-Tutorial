% !TeX root = ../../pythonTutorial.tex
\section{G�ltigkeitsbereich von Variablen}

Je nachdem, wie und wo Funktionen definiert sind, k�nnen diese zu anderen Ergebnissen f�hren. Zur Verdeutlichung folgt ein einfaches Beispiel (\ref{gueltigkeitsbereicheFunktionen:lst:simpleScope}).

\begin{lstlisting}[language=Python, label=gueltigkeitsbereicheFunktionen:lst:simpleScope]
# Beispiel zu G�ltigkeitsbereichen

def myFunction():
  # lokaler G�ltigkeitsereich der Funktion
  a = 1
  print('myFunction:', a)

# globaler G�ltigkeitsbereich
a = 0
myFunction()
print('global:', a)
\end{lstlisting}

Wenn wir diesen Code ausf�hren, sehen wir folgenden Output:

myFunction: 1
\\*
global: 0

In beiden Bereichen benutzen wir die Variable \lstinline$a$.
Die Funktion wird nach dem Initialisieren der Variable aufgerufen.
Warum erhalten wir also zwei unterschiedliche Werte?

Der Grund: es handelt sich nicht um die gleiche Variable, da beide Variablen in verschiedenen G�ltigkeitsbereichen definiert werden. Lie�en wir die lokale Zuweisung aus, w�rde zweimal der Wert 0 ausgegeben werden.
Python sucht nach dem n�chstm�glichen G�ltigkeitsbereich: \lstinline$lokal$, \lstinline$umschlie�end$, \lstinline$global$ und \lstinline$built-in$.

Nun ein Beispiel mit verschachtelten Funktionen:

\begin{lstlisting}[language=Python, label=gueltigkeitsbereicheFunktionen:lst:enclosingScope]
# Verschachtelte Funktionen

def enclosing():
  a = 1

  def innerFunction():
    a = 2
    print('innerste:', a)

  innerFunction()
  print('umschlie�end:', a)

a = 0
enclosing()
print('global:', a)
\end{lstlisting}

Nach dem Ausdrucken erhalten wir:

innerste: 2
\\*
umschlie�end: 1
\\*
global: 0


\subsection{Statements zu G�ltigkeitsbereichen - \mbox{global und nonlocal}}

Nicht nur durch die Positionen werden G�ltigkeitsbereiche definiert, auch durch die Schl�sselw�rter \lstinline$global$ und \lstinline$nonlocal$ k�nnen wir den G�ltigkeitsbereich bestimmen.

Durch \lstinline$nonlocal$ wird eine Variable auf die n�chst umschlie�ende Definition festgelegt (\ref{gueltigkeitsbereicheFunktionen:lst:nonlocalStatement}). 

\begin{lstlisting}[language=Python, label=gueltigkeitsbereicheFunktionen:lst:nonlocalStatement]
# Nonlocal Statement

def enclosing():
  a = 1

  def innerFunction():
    nonlocal a
    a = 2
    print('innerste:', a)

  innerFunction()
  print('umschlie�end:', a)

a = 0
enclosing()
print('global:', a)
\end{lstlisting}

Nach dem Ausdrucken erhalten wir:

innerste: 2
\\*
umschlie�end: 2
\\*
global: 0

Achtung: W�rde in der enclosing-Funktion \lstinline$a$ auf \lstinline$nonlocal$ gesetzt, dann k�me es zu einer \lstinline$Exception$, da die n�chste Ebene global ist.

Das Gleiche k�nnen wir mit dem globalen G�ltigkeitsbereich machen, wie in \ref{gueltigkeitsbereicheFunktionen:lst:globalStatement} gezeigt.

\begin{lstlisting}[language=Python, label=gueltigkeitsbereicheFunktionen:lst:globalStatement]
# Global Statement

def enclosing():
  a = 1

  def inner():
    global a
    a = 2
    print('innerste:', a)

  innereFunction()
  print('umschlie�end:', a)

a = 0
enclosing()
print('global:', a)
\end{lstlisting}

Nach dem Ausdrucken erhalten wir:

innerste: 2
\\*
umschlie�end: 1
\\*
global: 2


W�hrend \lstinline$nonlocal$ nur den n�chst umschlie�enden G�ltigkeitsbereich w�hlt -  in welcher die Variable deklariert wurde - greift \lstinline$global$ immer auf den globalen G�ltigkeitsbereich zu.

\newpage