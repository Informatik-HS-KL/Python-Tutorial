% !TeX root = ../../pythonTutorial.tex

\section{Klassen und Objekte}

Python ist wie Java eine objektorientierte Programmiersprache.
Das bedeutet, dass in Python fast alles aus Objekten und Klassen besteht.
Klassen sind Vorlagen, aus denen Objekte generiert werden k�nnen.
Dabei enth�lt die Klasse je nach Verwendungszweck Variablen und Methoden.
Ein Beispiel hierf�r w�re die Klasse \lstinline{Kreis} Jeder Kreis besitzt einen Radius, jedoch besitzt nicht jeder Kreis den selben.

\subsection{Klassen und Objekte erstellen}

Um dies Anhand eines Python Programms zu verdeutlichen, wird eine neue Klasse erstellt.
\begin{lstlisting}[language=Python,
label={classesandobjects:subsection:createclassesandobjects:createclass}]
# Erstellen einer Klasse
class Kreis:
radius = 1
\end{lstlisting}
Mithilfe der erstellten Klasse, k�nnen nun verschiedene Objekte erstellt werden, welche die Variablen und Methoden der Klasse beinhalten.
\begin{lstlisting}[language=Python, label={classesandobjects:subsection:createclassesandobjects:studentobject}]
class Kreis:
radius = 1
kreis1 = Kreis()
print(kreis1.radius)
\end{lstlisting}
Ausgabe des Programmcodes:
\begin{lstlisting}[language=Python, label={classesandobjects:subsection:createclassesandobjects:outputname}]
# Output der Konsole
1
\end{lstlisting}
Die Variablen der Objekte sind zun�chst gleich mit denen der Klasse, aus der diese erstellt wurde.
Allerdings sind die Variablen des erstellten Objekts unabh�ngig von denen der Klasse. Das bedeutet, dass diese auch unabh�ngig f�r jedes einzelne Objekt ge�ndert werden k�nnen. Die im obigen Beispiel verwendete Klasse ist in realen Anwendungen  nicht verwendbar, da die Attribute des Objekts von Anfang an festgelegt wurden. Um einen dynamischen Ansatz nutzen zu k�nnen, sollte man wie im folgenden Beispiel vorgehen.
\begin{lstlisting}[language=Python, label={classesandobjects:subsection:createclassesandobjects:selfinit}]
class Kreis:
def __init__(self, radius):
self.radius = radius

kreis1 = Kreis(3)
kreis2 = Kreis(5)
print(kreis1.radius)
print(kreis2.radius)
\end{lstlisting}
Ausgabe des Programmcodes:
\begin{lstlisting}[language=Python,
label={classesandobjects:subsection:initclass:outputtwonames}]
3
5
\end{lstlisting}

\subsection{Die init() Methode}
Jede Klasse hat eine \lstinline{init()} Methode, die immer ausgef�hrt wird, wenn die Klasse initiiert und ausgef�hrt wird. Diese Methode wird verwendet um den Variablen des Objektes einen Wert zu geben.
Hierbei ist der \lstinline{self} Parameter notwendig um die Klasse selbst zu referenzieren und um auf die Variablen der Klasse zugreifen zu k�nnen. Dieser Parameter ist also notwendig, muss aber nicht self genannt werden, sondern kann einen beliebigen Namen haben. Er wird als erster Parameter bei jeder Methode angegeben.
\begin{lstlisting}[language=Python,
label={classesandobjects:subsection:initclass:initclass}]
class Kreis:
def __init__(self, radius):
self.radius = radius

def getRadius(self):
print(self.radius)

kreis1 = Kreis(3)
kreis2 = Kreis(5)
print(kreis1.radius)
print(kreis2.radius)
\end{lstlisting}
Ausgabe des Programmcodes:
\begin{lstlisting}[language=Python,
label={classesandobjects:subsection:initclass:outputnames2}]
3
5
\end{lstlisting}
Objekte enthalten die aus den Klassen �bernommenen Methoden. Diese Methoden geh�ren jetzt zu dem Objekt. Um Parameter zu modifizieren oder zu l�schen, k�nnen folgende Befehle verwendet werden.
\begin{itemize}
	\item  \lstinline{objektname.parameter = neuer Wert (modifizieren)}
	\item \lstinline{del objektname.parameter (l�schen)}
\end{itemize}

\begin{lstlisting}[language=Python,
label={classesandobjects:subsection:initclass:delobjektname}]
class Kreis:
def __init__(self, radius):
self.radius = radius

def getRadius(self):
print(self.radius)

kreis1 = Kreis(3)
kreis2 = Kreis(5)
kreis1.radius = 3
kreis1.getRadius()
kreis2.getRadius()
del kreis2.radius
\end{lstlisting}
Ausgabe des Programmcodes:
\begin{lstlisting}[language=Python,
label={classesandobjects:subsection:initclass:outputthreenames}]
3
5
\end{lstlisting}

\subsection{Vererbung}
Ein weiterer wichtiger Aspekt in Python, ist die Vererbung und Erg�nzung einer Klasse.
\begin{lstlisting}[language=Python,
label={classesandobjects:subsection:heredity:heredity}]
class Kreis:
def __init__(self, radius):
self.radius = radius

def getRadius(self):
print(self.radius)

class Farbe:
def __init__(self, farbe):
Kreis.__init__(self, farbe)
self.farbe = "blau";

def getRadius(self):
print(Kreis.getRadius(self) + ", " + self.farbe)

kreis1 = Farbe(3,"rot")
kreis2 = Farbe(5, "gelb")
kreis1.getRadius()
kreis2.getRadius()
\end{lstlisting}
Ausgabe des Programmcodes:
\begin{lstlisting}[language=Python,
label={classesandobjects:subsection:heredity:heredityoutput}]
3, rot
5, gelb
\end{lstlisting}
Im Beispiel \ref{classesandobjects:subsection:heredity:heredity} wurde die Klasse Kreis durch die Klasse Farbe erweitert. Dies wird durch das Hinzuf�gen eines weiteren Attributes (bspw. farbe) realisiert. In dieser Klasse wurde die Methode  \lstinline{getRadius()} abge�ndert. Die Klasse, von der geerbt wird, kann entweder mit dem Klassennamen  \lstinline{Kreis} oder mit  \lstinline{super} referenziert werden. Trotz der �nderung der Methode  \lstinline{getRadius()} in der Klasse  \lstinline{Farbe}, beh�lt die Klasse  \lstinline{Kreis} ihre Methoden.

Daraufhin bleibt zu erw�hnen, dass auch komplette Objekte l�schbar sind. Dies ist mit dem Befehl  \lstinline{del Objektname} m�glich. Hierbei wird eine Fehlermeldung ausgegeben, da versucht wird, auf das gel�schte Objekt zuzugreifen.
\begin{lstlisting}[language=Python,
label={classesandobjects:subsection:heredity:heredity}]
class Kreis:
def __init__(self, radius):
self.radius = radius

def getRadius(self):
print(self.radius)

kreis1 = Kreis(3)
kreis2 = Kreis(5)
kreis1.radius = 6
kreis1.getRadius()
kreis2.getRadius()
del kreis2.radius
del kreis2
kreis2.getRadius()
\end{lstlisting}
Fehlermeldung beim Ausf�hren des Programmcodes:
\begin{lstlisting}[language=Python,
label={classesandobjects:subsection:heredity:errormessage}]
Traceback (most recent call last):
6
File"C:/users/Patrick/Desktop/python/objekte
_undklassen.py", line 15, in <module>
5
kreis2.getRadius()
NameError: name 'kreis2' is not defined
\end{lstlisting}

\uebung

\aufgabe{classesandobjects01}
\aufgabe{classesandobjects02}